\chapter{Introduction}
The aim of this thesis is to describe a theoretical model for the generation of entangled photons inside silicon optic chips. In particular, the structure I am going to study is composed by two microrings side coupled to two waveguides. Due to interference, the electric field inside a ring is enhanced and a great amount of energy is stored in the ring, which leads to non linear effects. Imposing a phase matching condition it is possible to select which non linear effect is more efficient and, therefore, which one is significant. Generation of entangled photons is possible by means of Spontaneous Four Wave Mixing, a third order non-linear effect of silicon.
In addition to this, the main advantage of using two coupled resonators is that it is possibile to change the phases of the fields in the waveguides between the rings, and I will show how this can be exploited in order to build a tunable device where we can manipulate the output state of the generated photons.
\\
In order to describe this process, the first chapter gives a description of the basic structures, microrings and waveguides and how we can describe their behaviour using a mathematical formalism. The second chapter is devoted to the explanation of the physics behind the phenomena and the mathematics which can be used to carry out the calculation. Finally in the last chapter the model and the results are presented.\\
The results obtained in this work are valid in general, however the numerical examples presented are based on a silicon structure.
\section{Integrated optics}
The microelectronics has been widely succesful and changed our daily life and work; nowadays microelectronics is present almost everywhere. The great success of these technologies is due to many reasons. Moore's law predicts that the number of transistors in a chip increases while their size decreases. The prediction still holds today, but we are reaching a saturation point: at the beginning, improvements were possible by increasing the clock frequencies of the processing units, but now the approach used is to increase the number of processing units. This kind of improvement faces several problems: the cores need to communicate with each other, so the problem is to realise efficient communication with high bandwidth and low power consumption. Photonics is a platform that can be used to achieve such goal: with light is possible to realize ultra-high speed switches, it has low power consumption, it doesn't have electromagnetic noise, it doesn't have Joule effect and since the wavelenght used is in the order of Thz, the bandwidth avaiable is much larger then in electronics, moreover it's possible to send signals over different wavelength at the same time. Several platforms have been developed with different media, such as silicon, lithium niobate, gallium arsenide and . Since the electronics industry is based on silicon, Silicon-on-Insulator (SOI) photonics is a platform to build optical chips. Indeed, the technology used for the fabrication of silicon chips is well studied and understood, it has reached a high level of sophistication and the industries and the infrastructures already exist. Futhermore silicon photonic is compatible with CMOS technology used in electronic chips, hence hybrid chips with electronics and photonics can be realised. Another important feature of silicon is its transparency at the important Telecom wavelengths (1300-1600 nm), which allows the fabrication of optic circuits with very low power losses. The aim of silicon photonics is to follow the philosophy behind microelectronics: small chips composed by few building blocks that can be used to build different devices changing the topology of the chip, and to realise these building blocks with as few materials as possible with a standard established among manifacturers.\\
From the point of view of quantum mechanics, light is composed by elementary particles called photons. Photons have quantum properties, hence it is possible to build quantum devices using the already implemented building blocks for silicon photonics. Photons can be used to represent quantum bits and due to their low decoherence effects, they are an attractive approch to quantum information processing. Quantum technologies can drastically improve some tasks in computation, measurement and communication , and quantum silicon photonics is a great platform for developing such technologies on a single chip. 


\section{Waveguides}
\begin{figure}[ht]
\centering
\includegraphics[width = .7\textwidth]{img/SOIstructure}
\caption{SOI chip layer structure}
\label{SOIstructure}
\end{figure}
\begin{figure}[ht]
\centering
\includegraphics[width = .5\textwidth]{img/TIRDiagram3}
\caption{Light behaviour at media boundary}
\label{criticalangle}
\end{figure}
A silicon optical chip is realised with three different layers as depicted in figure \ref{SOIstructure}. At the bottom we can find the substrate, a silicon layer which provides a base and a stable structure for the chip; just above it there is a layer made of silicon dioxide called cladding used to optical isolate the top layer from the substrate and finally at the top there is another silicon layer called the core. The thickness of the substrate is in the order of $700\, \mu m $ and the cladding in the order of some $\mu m$, while the core has a height in the order of a few hundred $nm$. The refractive index of silicon is $n_{Si} \simeq 3.5$, in the range of the Telecom wavelengths, while the refractive index of silicon dioxide is $n_{SiO_2}\simeq 1.54$, in the same range; this allows the confinement of light in the core layer by total internal refraction. Exactly like in optical fibers, when the light, travelling in the core layer, reaches the boundaries with the cladding or with the air (refractive index $\simeq 1$) is reflected back if the angle of incidence is greater than a critical angle. 
\\The most important component for an optical chip is a waveguide. The light signal is carried inside this waveguide by confinement of light. Consider the system in figure \ref{criticalangle}, when the light strikes the medium boundary, a part is reflected, while another part is transmitted. If the angle of incidence $\theta_I$ is less than a critical angle $\theta_c$, the light is completly reflected. Indeed, the angle of the refracted light is given by Snell's law
\begin{equation}n_1 \sin \theta_I = n_2 \sin \theta_T\end{equation}
which can be written as
\begin{equation}\sin \theta_I = \frac{n_1}{n_2}\sin \theta_R\end{equation}
if we impose $\theta_R$ to be at least $90\degree$, we obtain
\begin{equation}\sin \theta_I = \frac{n_1}{n_2} \implies \theta_I = \arcsin\left(\frac{n_1}{n_2}\right) \end{equation}
this is the critical angle; it is clear from the equation that the critical angle exists only if $n_1/n_2<1$, i.e. $n_1>n_2$. Hence the phenomenon of total internal reflection occurs only when light is inside a medium with a refractive index greater than the surrounding's. This is the case of a SOI device, where the light propagates inside a layer of silicon sandwiched between a layer of silica and the air, which both have a refractive index smaller than silicon's.\\
The geometrical optic description should be further elaborated for silicon waveguides. Indeed, the signal in the core layer can be best described with the electromagnetic wave theory. Inside a waveguide, several modes of propagation are possible, where electric field profile $E_m$ follows the Helmholtz equation \cite{book:saleh}:
\begin{equation}(\nabla_{xy}^2 + \beta_m^2)E_m(x,y)= \frac{\omega^2}{c^2}n^2E_m(x,y)\end{equation}
where $\beta_m$ is called the modal propagation costant, and is given by $\beta_m = \frac{\omega}{c}n_{eff}$ where $n_{eff}$ is the modal effective index. $m$ is an integer that represents the discrete mode of propagation.\\
An important physical consequence can be deduced from this equation: the tangential components of the electric and of the magnetic field cannot be discontinuous at the boundary, so the field is not totally inside the waveguide, but a small part can be found also outside. Assuming that a mode is confined in the waveguide, Maxwell's equations impose such boundary conditions on the electric and magnetic fields, the solution of these equations outside the waveguide cannot transport energy, otherwise the light would not be confined inside the waveguide. Therefore, the only solution is to have an exponentially decreasing mode represented by an evanescent wave. This evanescent wave explains why the effective index is present in the formula of the modal propagation costant, the wave, which travels inside the waveguide, propagates inside a medium with fixed refractive index and its tail propagates outside the waveguide where there is a different refractive index. The effective index accounts for this phenomenon.
Evanescent fields are also important when two waveguides are very close to each other: if light travels inside a waveguide and its evanescent tail reaches the neighbouring waveguide the light can penetrate into its core and there is a transfer of energy. This allows the coupling of different waveguides or, as we will see later, the coupling between a waveguide and a resonator. It is also interesting to look at the quantum description of this effect: a photon is described by a wave function which is a solution of the Schr{\"o}dinger equation, in the same mathematical way of Maxwell's equations, these impose the continuity of the wavefunction across the medium bondary. If two waveguides are close enough, the wave function is non zero inside both waveguides, so a photon have a non-zero probability to pass through the gap and change waveguide, in quantum mechanics this effect is called quantum tunneling. 
\section{Resonator}
\begin{figure}
\centering
\begin{subfigure}{0.5\textwidth}
\centering
\includegraphics[width = \textwidth]{img/APF}
\caption{}
\end{subfigure}%
\begin{subfigure}{0.5\textwidth}
\includegraphics[width = \textwidth]{img/ADF}
\caption{}
\end{subfigure}
\caption{Basic configuration with waveguides and a resonator (a) All Pass Filter (b) Add Drop Filter}\label{basicconfiguration}
\end{figure}

\begin{figure}
\centering
\includegraphics[width = .7\textwidth]{img/transferfunction}
\caption{Normalized transfer functions $H^T_{AD}$ (black line) and $H^D_{AD}$ (red line). Moreover some parameters are showed: Free Spectral Range (FSR), the Extinction ratio (ER) for the Through and for the Drop port, and the resonance FWHM.}\label{transfer}
\end{figure}

A waveguide can be bent and closed onto itself to provide coherent feedback to the circulating light. This device is called a resonator and it can have different shape, such as a ring or a racetrack. Due to the size of these devices they are called microresonators. For ease of calculation we will treat only microrings with fixed radius. Inside a ring resonator light interferes constructively if the equation $n_{eff} L = m \lambda_m$ is satisfied, where L is the circumference, $m$ is an integer and $\lambda_m$ is the light wavelength. A consequence of constructive interference is that inside the resonator a large amount of energy is stored and the intesity of the field is enhanced. A strong electric field can cause non-linear effects. This is a reason for the importance of resonators: with low power input it is possible to have non-linear effects that usually require a more powerful source. Ring resonators have many applications: for example they can be used as filters for specific wavelengths for multiplexing applications, sensing, signal modulation and active devices for building integrated microlasers.\\
There are two main ways to create circuits with the two basic blocks just described: the first one is a waveguide side coupled to a single ring, and this configuration is called All Pass Filter (APF), while if there are two waveguides coupled to a single ring, the configuration is called Add-Drop Filter (ADF). In figure \ref{basicconfiguration} we can see both configurations. The coupling is possible by means of the evanescent wave, since the gap between the waveguide and the ring is very narrow (on the order of $\simeq 100\, nm$).\\
Coupling can be seen as a quad port beam splitter and the relationship between the complex amplitudes of waves in input and output can be represented by the following matrix
\begin{equation}M = \begin{pmatrix}
r & ik \\
ik & r\\
\end{pmatrix}\end{equation}
where $r$ is the reflection coefficient and $k$ is the transmission coefficient, they satisfy $r^2 +k^2 = 1$. The elements of the matrix can be found by imposing energy conservation among input and output \cite{thesis:masi}. Let us focus now on the APF configuration: using the above matrix we want to find the relation bewteen $a$ and $c$ in order to find the transfer function of the device
\begin{equation}\label{APF}\begin{pmatrix}c \\ d \end{pmatrix} = M \begin{pmatrix}a\\b \end{pmatrix}\end{equation}
$b$ and $d$ are connected with the roundtrip phase condition $b = e^{-\alpha 2\pi R} e^{-i\beta 2\pi R}d \equiv\tau e^{-i\phi(\lambda)}d $, where $\alpha$ is the linear loss coefficient, $R$ the ring radius and $\beta$ is the resonance wavevector $\beta = \frac{2\pi n_{eff}}{\lambda}$. Solving the system \eqref{APF} leads to the transfer function
\begin{equation}H_{AP} = \frac{c}{a} = \frac{\tau - re^{i\phi(\lambda)}}{r\tau -e^{i\phi(\lambda)}}\end{equation}
For the ADF configuration the expression of the transfer function can be obtained in a similar way, but now we need to handle two beam splitters, and the round trip phase condition is more complicated: $f = e^{-\alpha \pi R} e^{-i\beta \pi R}d$ and $b=e^{-\alpha \pi R} e^{-i\beta \pi R}h$, working out the calculation we get the final result
\begin{equation}H^T_{AD} = \frac{c}{a} = \frac{k^2\sqrt{\tau} e^{i\phi/2}}{r^2\tau -e^{i\phi}}\qquad H^D_{AD} = \frac{g}{a} = \frac{r(e^{i\phi} - \tau)}{e^{i\phi}-r^2\tau} \end{equation}
the plot of these transfer functions is in figure \ref{transfer}, where some quantities are also depicted. Another important quantity is the quality factor $Q = \frac{\lambda}{\Delta \lambda}$, where $\Delta \lambda$ is the full width at half maximum (FWHM) of the lorentzian resonance at wavelength $\lambda$. Equivalently the quality factor can be seen as the ratio between the stored energy inside the cavity and the amount lost per cycle. Furthermore, the quality factor is directly correlated with the enhancement factor $EF$ which is the ratio between the amplitude of the electric field inside the resonator and the exciting electric field. An expression for the quality factor can be obtained by taking two successive resonances and calculating the free spectral range, with a Taylor expansion it can be found \cite{thesis:borghi} for the APF and ADF configurations
\begin{equation}
Q_{APF} = \frac{\pi n_g \lambda_m 2\pi R \sqrt{r\tau}}{(1-r\tau)\lambda_m} \qquad Q_{ADF} = \frac{\pi n_g \lambda_m 2\pi R r\sqrt{\tau}}{(1-r^2\tau)\lambda_m}
\end{equation}
where $n_g$ is the group index. From this equation we can see that it is possible to achieve high quality factor with large rings and low losses. For example with losses in the order of $3 \frac{dB}{\text{cm}}$, a radius of $10\, \mu m$, with an input of $1550$ nm and $k = 0.03$, the quality factor is approximately $10^4$.
\section{Coupled resonators}\label{coupled}
\begin{figure}[H]
\centering
\includegraphics[width = .7\textwidth]{img/coupled}
\caption{Two resonators coupled together}
\label{resonatorcoupled}
\end{figure}
A more interesting configuration of microresonators is presented in figure \ref{resonatorcoupled}. The configuration consists of by two ADF connected in series and it is the configuration on which this work is based. Between the microresonators there is a heater that by heating up the waveguide is able to change the phase of the field travelling inside it. This is a conseguence of the thermo optic effect by which the refractive index depends on the temperature \cite{thesis:borghi}. The idea is that the field enhancement leads to non-linear effects, among which there is spontaneous four wave mixing that can be exploited to generate photon pairs. If the input laser is in A, the generated photons can exit both from B, both from C or one photon in B and one photon in C, but, by changing the phase $\phi_1$ and $\phi_2$, it is possible to decide where the photons exit. A useful quantity that will be needed in this work is the field enhancement of the rings. The analytic expression can be found using the formalism developed in the previous section, and the results are
\[FE_1 = \frac{ie^{i(\varphi_1+\phi_2+\phi_1)}k (e^{i\varphi_2}-r^2\tau_2)-ie^{\frac{i}{2}(\varphi_1+\varphi_2)}k^3(k^2+r^2)\sqrt{\tau_1\tau_2}\tau L^2}{e^{i(\phi_1+\phi_2)}(e^{i\varphi_1}-r^2\tau_1)(e^{i\varphi_2}-r^2\tau_2)-e^{\frac{i}{2}(\varphi_1+\varphi_2)}k^4\sqrt{\tau_1\tau_2}\tau L^2}\]
\[FE_2 = \frac{ie^{i(\varphi_2+\phi_1)}kr (-e^{i\varphi_1}+(r^2+k^2)\tau_1)\tau L}{e^{i(\phi_1+\phi_2)}(e^{i\varphi_1}-r^2\tau_1)(-e^{i\varphi_2}+r^2\tau_2)+e^{\frac{i}{2}(\varphi_1+\varphi_2)}k^4\sqrt{\tau_1\tau_2}\tau L^2}\]
where $FE_1$ refers to the ring on the left and $FE_2$ to the ring on the right, $\varphi = \beta L$ and $L=2\pi R$. We stress that the phases $\phi_1$ and $\phi_2$ refer to the phases induced by the heaters between the rings, while $\varphi_1$ and $\varphi_2$ are the phases inside the rings.\\
 Figure \ref{FE} shows field enhancements in the case of $\phi_1 + \phi_2 = 2m\pi$ around a single ring resonance wavelength. The condition $\phi_1 + \phi_2 = 2m\pi$ is important because it enhances the optical power inside the ring and, as can be seen from the figure, split the energy equally between the two rings. An equally split energy means that the two rings are indistinguishable and therefore there is the same probability that the photons are generated inside one or the other ring. Indeed, if the energy is not split equally, it means that only one ring works and the configuration is identical to a simple ADF.


\begin{figure}
\centering
\includegraphics[width = .8\textwidth]{img/FE_fase_2pi}
\caption{Field enhancements with $\phi_1+\phi_2 = 2\pi$, ring 1 refers to the left one and 2 to the right one}
\label{FE}
\end{figure}

\begin{figure}
\centering
\includegraphics[width = .8\textwidth]{img/FE_fase_pi}
\caption{Field enhancements with $\phi_1+\phi_2 = -\pi$, ring 1 refers to the left one and 2 to the right one}
\label{FE}
\end{figure}